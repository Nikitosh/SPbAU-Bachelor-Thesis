\documentclass[10pt]{beamer}

\mode<presentation> {

% \usetheme{Berlin}  % Squares
\usetheme{Madrid}  % Circles & dense
% \usetheme{Frankfurt}  % Cirles

\setbeamertemplate{headline}{%
\leavevmode%
  \hbox{%
    \begin{beamercolorbox}[wd=\paperwidth,ht=2.5ex,dp=1.125ex]{palette quaternary}%
    \insertsectionnavigationhorizontal{\paperwidth}{}{\hskip0pt plus1filll}
    \end{beamercolorbox}%
  }
}

\setbeamertemplate{navigation symbols}{}  % To remove the navigation symbols from the bottom of all slides uncomment this line
\setbeamertemplate{items}[circle]

% \useoutertheme{miniframes}
% \useinnertheme{circles}

}

\usepackage{graphicx}
\usepackage{booktabs}
\usepackage{fontspec}
\usepackage{xunicode}
\usepackage{tabularx}
\usepackage{xltxtra}
\usepackage{xecyr}
\usepackage{hyperref}
\usepackage{amsthm}
\usepackage{blindtext}
\usepackage{color}
\usepackage[normalem]{ulem}
\usepackage{url}
\usepackage[font=small,labelfont=bf]{caption}
\usepackage[subrefformat=parens]{subcaption}
\captionsetup{compatibility=false}
\captionsetup[subfigure]{labelformat=empty}
\captionsetup[figure]{labelformat=empty}

\usepackage{polyglossia}
\setdefaultlanguage{russian}
\setmainfont[Mapping=tex-text]{CMU Serif}
\setsansfont[Mapping=tex-text]{CMU Sans Serif}
\setmonofont[Mapping=tex-text]{CMU Serif}

\makeatletter
\DeclareUrlCommand\ULurl@@{%
  \def\UrlFont{\ttfamily\color{blue}}%
  \def\UrlLeft{\uline\bgroup}%
  \def\UrlRight{\egroup}}
\def\ULurl@#1{\hyper@linkurl{\ULurl@@{#1}}{#1}}
\DeclareRobustCommand*\ULurl{\hyper@normalise\ULurl@}
\makeatother

\newcommand\TODO[1]{\textcolor{red}{{\Large TODO: #1}}}
\newcommand\NaN{\textcolor{red}{NaN}}

%-------------------------------------------------------------------------------
%	TITLE PAGE
%-------------------------------------------------------------------------------

\title[Определение наилучшего ответа на SO]{Определение наилучшего ответа на StackOverflow}
\author[Никита Подгузов]{
Никита Подгузов 
\texorpdfstring{\vskip0.5cm \footnotesize{Научный руководитель: Рауф Курбанов}}{}
}
\institute[СПбАУ]{Санкт-Петербургский Академический университет}
\date{18 июня 2018 года}
\begin{document}

%-------------------------------------------------------------------------------
%	PRESENTATION SLIDES
%-------------------------------------------------------------------------------

\begin{frame}
\titlepage
\end{frame}

%-------------------------------------------------------------------------------

\begin{frame}
\frametitle{Введение}
\framesubtitle{Обзор}

\begin{columns}
    \begin{column}{0.6\textwidth}
        Возможности сервисов вопросов и ответов:
        \begin{itemize}
            \item Задавать вопрос (и отмечать правильный ответ) 
            \item Отвечать на вопросы, заданные другими пользователями
            \item Голосовать за понравившиеся ответы
        \end{itemize}
    \end{column}
    \begin{column}{0.4\textwidth}
        \begin{center}
            \includegraphics[width=0.7\textwidth]{images/yahoo_answers_logo.png} \\
            \includegraphics[width=0.7\textwidth]{images/quora_logo.png} \\
            \includegraphics[width=0.7\textwidth]{images/stackoverflow_logo.png}
        \end{center}
    \end{column}
\end{columns}
\end{frame}

%-------------------------------------------------------------------------------

\begin{frame}
\frametitle{Введение}
\framesubtitle{ServerFault}

\begin{columns}
    \begin{column}{0.4\textwidth}
    	Особенности системы:
    	\begin{itemize}
			\item Узкоспециализированная
			\item Большая база вопросов
    	\end{itemize}
    \end{column}
    \begin{column}{0.6\textwidth}
        \begin{center}
            \includegraphics[width=0.82\textwidth]{images/serverfault_screen.png}
        \end{center}
    \end{column}
\end{columns}

\end{frame}

%-------------------------------------------------------------------------------

\begin{frame}
\frametitle{Введение}
\framesubtitle{Постановка задачи}

Проблемы:
\begin{itemize}
	\item Большая доля <<неразрешенных>> вопросов
	\item Нет возможности помочь оценить правильность ответов пользователю, задавшему новый вопрос
\end{itemize}

\vskip0.5cm

Хотим научиться определять правильные ответы, используя базу вопросов ServerFault

\end{frame}

%-------------------------------------------------------------------------------

\begin{frame}
\frametitle{Актуальность}
\framesubtitle{Google \& StackOverflow}

\begin{figure}
	\begin{subfigure}[b]{0.49\linewidth}
		\centering
		\includegraphics[width=\linewidth]{images/google_search_old.png}
		\caption{Старая версия}
	\end{subfigure}\hfill
	\begin{subfigure}[b]{0.49\linewidth}
		\centering
		\includegraphics[width=\linewidth]{images/google_search_new.png}
		\caption{Новая версия}
	\end{subfigure}
\end{figure}

\end{frame}

%-------------------------------------------------------------------------------

\begin{frame}
\frametitle{Введение}
\framesubtitle{Обзор имеющихся решений}

\begin{itemize}
	\setlength{\itemsep}{1em}

	\item Burel et al. (2012)
	
	<<Automatic Identification of Best Answers in Online Enquiry Communities>> 

	\item Tian et al. (2013) 
	
	<<Towards Predicting the Best Answers in Community-Based Question-Answering Services>> 

	\item Gkotsis et al. (2014)
	
	<<It’s all in the Content: State of the art Best Answer Prediction based on Discretisation of Shallow Linguistic Features>>

\end{itemize}

\end{frame}

%-------------------------------------------------------------------------------

\begin{frame}
\frametitle{Введение}
\framesubtitle{Обзор имеющихся решений}

\begin{itemize}
	\item Тестирование проводилось на данных ServerFault
	\item Виды фичей: $A \leftrightarrow A$, $A \leftrightarrow Q$, $A$, $user/answer-rating$, $thread$ 
	\item Лингвистических фичи (длина текста, количество предложений, читаемость и др.)
	\item Vector Space Model + TF-IDF для определения похожести
	\item Вероятностная униграмная модель для оценки вероятности ответа
	\item Группировка ответов и дискретизация фичей
	\item Не учитывается наличие сниппетов кода
	\item Alternating Decision Tree / Random Forest Classifier
\end{itemize}

\end{frame}

%-------------------------------------------------------------------------------

\begin{frame}
\frametitle{Введение}
\framesubtitle{Минусы имеющихся решений}

\begin{itemize}
	\setlength{\itemsep}{1em}
	\item Не используется текст вопроса
	\item Не учитывается порядок слов в предложении и их контекст
	\item Игнорируются семантические свойства слов
\end{itemize}

\end{frame}


%-------------------------------------------------------------------------------

\begin{frame}
\frametitle{Цель и задачи}

Цель: Научиться определять правильность ответа на StackOverflow, используя тексты вопросов и ответов

\medskip

Задачи:
\begin{itemize}
	\item Произвести сбор данных для обучения и тестирования
	\item Разработать несколько моделей классификаторов на основе нейронных сетей, использующих содержания вопроса и ответа
	\item Сравнить результаты с имеющимися работами
\end{itemize}


\end{frame}

%-------------------------------------------------------------------------------

\begin{frame}
\frametitle{Данные}
\framesubtitle{Общие факты}

Данные:

\begin{itemize}
	\item База вопросов ServerFault
	\item \texttt{XML}-файл размером $\sim0.9GB$
	\item Новые данные могут быть получены с помощью \texttt{API}
	\item Формат файла: $type\_id$, $id$, $score$, $date$, $body$
\end{itemize}

\end{frame}

%-------------------------------------------------------------------------------

\begin{frame}
\frametitle{Данные}
\framesubtitle{Анализ и обработка}

Анализ:

\begin{itemize}
	\item $684$ тысячи постов, из них $257$ тысяч вопросов и $427$ тысяч ответов
	\item $130$ тысяч вопросов $(51\%)$ без отмеченного правильного ответа
	\item $28$ тысяч вопросов $(11\%)$, у которых нет ни одного ответа
	\item Большое количество технических терминов и слов с опечатками
\end{itemize}

После обработки получили $80$ тысяч вопросов и $180$ тысяч ответов

\end{frame}

%-------------------------------------------------------------------------------

\begin{frame}
\frametitle{Данные}
\framesubtitle{Извлечение признаков}

Признаки:

\begin{itemize}
	\item Векторные представления вопроса и ответа
	\item Лингвистические: количество ссылок, параграфов, фрагментов кода, длина текста, средняя длина предложения, различные индексы удобочитаемости
	\item Словарные: различные модификации коэффициента Жаккара для текстов вопроса и ответа
	\item Метаинформация: дата публикации и позиция ответа
\end{itemize}

\end{frame}

%-------------------------------------------------------------------------------

\begin{frame}
\frametitle{Подходы к представлению текста}

\begin{columns}
    \begin{column}{0.5\textwidth}
		\begin{itemize}
			\item
			Bag of words

			Минусы: не учитывается порядок слов и семантика, большая размерность

			\item
			Word2Vec
	
			Обучение на неразмеченном корпусе текстов, для каждого слова получаем вектор, отражающий его семантику.

			Минусы: не дает векторное представления для слов не из словаря

			\item
			Fasttext

			Модификация Word2Vec, основанная на работе с $n$-грамммами букв
		\end{itemize}
	\end{column}
    \begin{column}{0.5\textwidth}
        \begin{center}
        	\begin{figure}
	            \includegraphics[width=0.9\textwidth]{images/cbow.png} 
    	        \caption{Архитектура модели CBOW}
        	\end{figure}
        \end{center}
    \end{column}
\end{columns}

\end{frame}

%-------------------------------------------------------------------------------

\begin{frame}
\frametitle{Решение задачи}
\framesubtitle{Сверточная нейронная сеть}

\begin{figure}
	\centering
	\includegraphics[width=0.75\linewidth]{images/my_cnn.png}
\end{figure}

\end{frame}

%-------------------------------------------------------------------------------

\begin{frame}
\frametitle{Решение задачи}
\framesubtitle{Рекуррентная нейронная сеть}

\begin{figure}
	\centering
	\includegraphics[width=0.53\linewidth]{images/my_rnn.png}
\end{figure}

\end{frame}

%-------------------------------------------------------------------------------

\begin{frame}
\frametitle{Результаты}

\begin{center}
	\small
    \begin{tabularx}{\textwidth}{| l | X | l | l | l | l | l |}
        \hline
	    \textbf{Модель} & \textbf{Виды признаков} & $\mathbf{Acc}$ & $\mathbf{P}$ & $\mathbf{R}$ & $\mathbf{F_1}$ & $\mathbf{AUC}$ \\ \hline
        \hline
	    First-answer & метаинформация & 0.66 & 0.69 & 0.60 & 0.64 & 0.66 \\ \hline
    	Naive-Bayes with TF-IDF & словарные & 0.6 & 0.58 & 0.54 & 0.56 & 0.59 \\ \hline
        \hline
	    Burel et al. (2012) & лингвистические, словарные, метаинформация & - & 0.77 & 0.77 & 0.76 & 0.83 \\ \hline
    	Tian et al. (2013) & лингвистические, метаинформация & 0.72 & - & - & - & - \\ \hline
        Gkotsis et al. (2014) & лингвистические, словарные, метаинформация & - & 0.83 & 0.66 & 0.74 & 0.85 \\ \hline
	    \hline
        CNN & текстовые, лингвистические, словарные, метаинформация & \textbf{0.77} & 0.84 & 0.61 & 0.71 & \textbf{0.86} \\ \hline
	    RNN & текстовые, лингвистические, словарные, метаинформация & \textbf{0.79} & 0.87 & 0.62 & 0.73 & \textbf{0.88} \\ \hline
    \end{tabularx}
\end{center}
\end{frame}

%-------------------------------------------------------------------------------

\begin{frame}
\frametitle{Выводы}

\begin{itemize}
	\item Подготовлен корпус данных, представляющий из себя ответы на вопросы с сайта Server Fault
	\item Разработано и реализовано несколько различных архитектур нейронных сетей для решения задачи определения наилучшего ответа на StackOverflow
	\item Показано, что текст вопроса является важным признаком для классификации
	\item Полученные результаты свидетельствуют о том, что нейронные сети справляются с задачей лучше методов, использованных в других статьях
\end{itemize}

\end{frame}


%-------------------------------------------------------------------------------
%	END PRESENTATION SLIDES
%-------------------------------------------------------------------------------

\end{document}