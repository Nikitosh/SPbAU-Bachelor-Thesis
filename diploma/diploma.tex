% В этом шаблоне используется класс spbau-diploma. Его можно найти и, если требуется,
% поправить в файле spbau-diploma.cls
\documentclass{spbau-diploma}
\usepackage{wasysym}
\usepackage{subfiles}
\usepackage{enumitem}
\usepackage{graphicx}
\usepackage{svg}
\usepackage{tabularx}
\usepackage{caption}
\usepackage{subcaption}
% \usepackage[utf8]{inputenc}
% \usepackage[english]{babel}
 
% \usepackage{amsthm}

\newtheorem{theorem}{Теорема}[section]
\newtheorem{remark}{Замечание}[section]
\newtheorem{corollary}{Следствие}[section]
\newtheorem{lemma}[theorem]{Лемма}
\newtheorem{definition}{Определение}[section]
\newtheorem{assumption}{Предположение}
\newtheorem{observation}{Наблюдение}

% \usepackage{titlesec}
% \newcommand{\sectionbreak}{\clearpage}

\begin{document}
% Год, город, название университета и факультета предопределены,
% но можно и поменять.
% Если англоязычная титульная страница не нужна, то ее можно просто удалить.
\filltitle{ru}{
    chair              = {Кафедра математических и информационных технологий},
    title              = {Определение наилучшего ответа на StackOverflow},
    % Здесь указывается тип работы. Возможные значения:
    %   coursework - Курсовая работа
    %   diploma - Диплом специалиста
    %   master - Диплом магистра
    %   bachelor - Диплом бакалавра
    type               = {bachelor},
    position           = {студента},
    group              = 402,
    author             = {Подгузов Никита Владимирович},
    supervisorPosition = {},
    supervisor         = {Курбанов Р. Э.},
    reviewerPosition   = {},
    reviewer           = {Артамонов А. С.},
    chairHeadPosition  = {д.\,ф.-м.\,н., профессор},
    chairHead          = {Омельченко А.\,В.},
    % university       = {САНКТ-ПЕТЕРБУРГСКИЙ АКАДЕМИЧЕСКИЙ УНИВЕРСИТЕТ},
    % faculty          = {Центр высшего образования},
    % city             = {Санкт-Петербург},
    % year             = {2018}
}
\filltitle{en}{
    chair              = {Department of Mathematics and Information Technology},
    title              = {Best answer detection on StackOverflow},
    author             = {Nikita Podguzov},
    supervisorPosition = {},
    supervisor         = {Rauf Kurbanov},
    reviewerPosition   = {},
    reviewer           = {Alexey Artamonov},
    chairHeadPosition  = {professor},
    chairHead          = {Alexander Omelchenko},
}
\maketitle
\tableofcontents

\section*{Введение}
\subfile{sections/0_introduction}

\section{Обзор предметной области}
\subfile{sections/1_subject_area}

\section{Цель и задачи}
\subfile{sections/2_goals}

\section{Описание данных}
\subfile{sections/3_data}

\section{Методы и реализация}
\subfile{sections/4_methods_and_implementation}

\clearpage
\section{Тестирование}
\subfile{sections/5_experiments}

\section*{Заключение}
\subfile{sections/6_conclusion}

\bibliographystyle{ugost2008l}
%\bibliographystyle{unsrt}
\bibliography{diploma}
\end{document}
