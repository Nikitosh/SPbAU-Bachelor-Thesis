% В этом шаблоне используется класс spbau-diploma. Его можно найти и, если требуется,n
% поправить в файле spbau-diploma.cls
\documentclass{spbau-diploma}
\usepackage{wasysym}
\usepackage{subfiles}
\usepackage{enumitem}

% \usepackage[utf8]{inputenc}
% \usepackage[english]{babel}
 
% \usepackage{amsthm}

\newtheorem{theorem}{Теорема}[section]
\newtheorem{remark}{Замечание}[section]
\newtheorem{corollary}{Следствие}[section]
\newtheorem{lemma}[theorem]{Лемма}
\newtheorem{definition}{Определение}[section]
\newtheorem{assumption}{Предположение}
\newtheorem{observation}{Наблюдение}

% \usepackage{titlesec}
% \newcommand{\sectionbreak}{\clearpage}

\begin{document}
% Год, город, название университета и факультета предопределены,
% но можно и поменять.
% Если англоязычная титульная страница не нужна, то ее можно просто удалить.
\filltitle{ru}{
    chair              = {Кафедра математических и информационных технологий},
    title              = {Определение правильного ответа на StackOverflow},
    % Здесь указывается тип работы. Возможные значения:
    %   coursework - Курсовая работа
    %   diploma - Диплом специалиста
    %   master - Диплом магистра
    %   bachelor - Диплом бакалавра
    type               = {bachelor},
    position           = {студента},
    group              = 402,
    author             = {Подгузов Никита Владимирович},
    supervisorPosition = {},
    supervisor         = {Курбанов Р.},
    reviewerPosition   = {},
    reviewer           = {-},
    chairHeadPosition  = {д.\,ф.-м.\,н., профессор},
    chairHead          = {Омельченко А.\,В.},
    % university       = {САНКТ-ПЕТЕРБУРГСКИЙ АКАДЕМИЧЕСКИЙ УНИВЕРСИТЕТ},
    % faculty          = {Центр высшего образования},
    % city             = {Санкт-Петербург},
    % year             = {2018}
}
\filltitle{en}{
    chair              = {Department of Mathematics and Information Technology},
    title              = {Correct answer detection on StackOverflow},
    author             = {Nikita Podguzov},
    supervisorPosition = {},
    supervisor         = {Rauf Kurbanov},
    reviewerPosition   = {},
    reviewer           = {-},
    chairHeadPosition  = {professor},
    chairHead          = {Alexander Omelchenko},
}
\maketitle
\tableofcontents

\section*{Введение}
\subfile{sections/introduction}

\subsection*{Мотивация}
\label{sec:motivation}
\subfile{sections/motivation}

\section{Обзор решений}
\label{sec:existing_solutions}
\subfile{sections/existing_solutions}

\section{Основные цели}
\label{sec:goals}
\subfile{sections/goals}

\section{Методы и реализация}

\subsection{Описание данных}
\label{sec:data}
\subfile{sections/data}

\label{sec:implementation}
\subfile{sections/implementation}

\newpage
\section{Обсуждение результатов}
\label{sec:experiments}
\subfile{sections/experiments}


\section{Заключение}
\label{sec:results}
\subfile{sections/results}

\bibliographystyle{ugost2008ls}
\bibliography{diploma}
\end{document}
