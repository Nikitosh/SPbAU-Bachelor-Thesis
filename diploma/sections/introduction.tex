\documentclass[../diploma.tex]{subfiles}

\begin{document}
        
    \label{sections/introduction}
    
    С распространением и развитием интернета стали популярны системы вопросов и ответов --- вид сайтов, 
    позволяющий пользователям задавать вопросы и отвечать на вопросы, заданные другими пользователями.
    Среди наиболее популярных подобных систем можно выделить многопрофильные (такие, как \textit{Yahoo Answers}, \textit{Quora}) 
    и узкоспециализированные (например, \textit{Stackoverflow}).
    В рамках данной работы мы сконцентрируемся именно на системе Stackoverflow, 
    которая в первую очередь является системой для вопросов по программированию и смежным областям.

    На Stackoveflow у пользователя есть возможность после того, как он задал интересующий его вопрос, отметить ответ, решивший его проблему, как правильный.
    Несмотря на это, достаточно большая часть вопросов остается без подобной отметки, что может усложнить поиск нужной информации по данному вопросу другим пользователям. \\

    *Сюда можно добавить скриншот вопроса с StackOverflow с интересующими нас частями веб-страницы*\\

    Таким образом появляется задача определения правильного ответа для вопросов с Stackoverflow.

    В связи с многократно возросшими вычислительными мощностями в последнее время в области машинного обучения приобрели популярность нейронные сети.
    Сфера применимости моделей, использующих нейронные сети, очень широка: от распознавания образов на изображениях до предсказания курсов валют на бирже.
    Несмотря на наличие работ, исследующих задачу предсказания лучшего ответа, в основном в них используются более классические методы машинного обучения,
    хотя нейронные сети и показывают достойные результаты в задачах классификации текстов.

    В данной работе исследуется применимость модели, основанной на нейронных сетях, к поставленной задаче, 
    а также сравниваются ее модификации, использующие специфику задачи: технического языка вопросов, а также наличие сниппетов кода в вопросах и ответах.

\end{document}

