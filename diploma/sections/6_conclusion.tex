\documentclass[../diploma.tex]{subfiles}
 
\begin{document}

	\label{sec:conclusion}

	В рамках данной работы были достигнуты следующие результаты:

	\begin{itemize}
		\item
		Рассмотрены общие подходы к задаче классификации текстов и существующие решения для поставленной в данной работе задачи, проанализированы их достоинства и недостатки.
		                                                                                                                      			
		\item
		Подготовлен корпус данных, представляющий из себя ответы на вопросы с сайта Server Fault.

		\item
		Предложены и посчитаны дополнительные признаки, которые в дальнейшем используются для классификации.

		\item
		Разработано несколько моделей бинарных классификаторов на основе рекуррентных и сверточных нейронных сетей.

		\item
		Произведено тестирование и сравнение полученных результатов с имеющимися в области исследованиями.

	\end{itemize}

	Представленная модель использует идеи рекуррентных нейронных сетей для анализа содержания вопросов и ответов, при этом 
	комбинирует в себе как семантические признаки, так и более классические лингвистические и словарные.
	 
	В итоге, в задаче классификации наилучших ответов наш метод позволяет достичь результатов, 
	которые сравнимы по одним метрикам, либо превосходят по другим уже имеющиеся исследования в этой области.

	В качестве дальнейшего развития имеющихся моделей можно выделить следующие направления:

	\begin{itemize}
		\item
		Использование более сложной архитектуры для определения того, решает ли ответ обозначенную в вопросе проблему или нет: например, 
		совместное использование рекуррентных и сверточных слоев, как в работе \cite{article:text_rnn_cnn}. 

		%\item
		%Как уже было отмечено ранее, в связи с технической специализацией вопросов в текстах присутствует множество терминов, 
		%поэтому еще одним направлением развития может стать выделение подобных сущностей и отношений между ними \cite{article:entity_recognition} и работе с ними отдельно.

		%В частности, это может помочь понять, насколько ответы, описывающие какой-либо процесс действительно соответствуют тематике вопроса.

		\item
		Если пытаться расширить имеющуюся модель на основной сайт StackOverflow, то стоит также учитывать содержание фрагментов кода, 
		которые присутствуют более, чем в половине постов.
		Для этого может быть применен механизм векторизации для кода \cite{article:code_embedding}, который имеет что-то общее с принципом работы Word2Vec.

	\end{itemize}
	
\end{document}

