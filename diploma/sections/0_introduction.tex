\documentclass[../diploma.tex]{subfiles}

\begin{document}
        
	\label{sec:introduction}

    С распространением и развитием интернета стали популярны системы вопросов и ответов~--- вид сайтов, 
    позволяющий пользователям задавать вопросы и отвечать на вопросы, заданные другими пользователями.
    Среди наиболее популярных подобных систем можно выделить многопрофильные (такие как \textit{Yahoo Answers}, \textit{Quora}) 
    и узкоспециализированные (например, \textit{StackOverflow} \cite{online:stackoverflow}).
    В рамках данной работы мы сконцентрируемся на системе StackOverflow, 
    которая в первую очередь специализируется на вопросах по программированию и смежным областям.

    На сайтах системы StackOverflow у автора есть возможность после того, как он задал интересующий его вопрос, отметить ответ, решивший его проблему, как правильный.
    
    Несмотря на это, достаточно большая часть вопросов остается без подобной отметки даже при наличии хороших ответов-кандидатов, 
    что может усложнить поиск нужной информации по данному вопросу другим пользователям. 
    Кроме того, пользователю, задавшему вопрос по какой-то не самой популярной теме, может быть сложно оценить правильность и релевантность полученных ответов.

    Таким образом, перед нами появляется задача определения правильного ответа для вопросов со StackOverflow.
       
    \subsection*{Мотивация}
	
	В связи с многократно возросшими вычислительными мощностями в последнее время в области машинного обучения приобрели популярность нейронные сети.
    Сфера применимости моделей, использующих нейронные сети, очень широка: 
    от распознавания образов на изображениях \cite{article:cnn} до предсказания курсов валют на бирже \cite{article:currency}.
    В частности, многие современные системы, связанные с обработкой естественного языка, 
    например, платформы машинного перевода \cite{article:nmt} или чатботы \cite{article:chatbot}, используют нейронные сети.

    Несмотря на наличие работ, исследующих задачу предсказания лучшего ответа для систем вопросов и ответов, 
    в основном в них используются более классические методы машинного обучения,
    хотя нейронные сети и показывают достойные результаты в задачах классификации текстов.

    В данной работе исследуется применимость нескольких моделей, основанных на нейронных сетях и использующих такие особенности сайта StackOverflow, 
    как технический язык вопросов и наличие фрагментов кода в вопросах и ответах, к поставленной задаче.

	Также хочется подчеркнуть, что данная работа ставит своей целью определение правильного ответа на основе текстовых признаков и 
	без использования какой-либо внешней информации о рейтинге, то есть используя только содержание и дату публикации вопросов и ответов.
	Это делает подобную систему более универсальной: во-первых, она абстрагируется от конкретной системы рейтингов и репутации, 
	а во-вторых, позволяет отвечать на новые или непопулярные вопросы, в которых подобная информация может быть недоступна.
	Поэтому в рамках данной работы не рассматриваются рейтинги ответов или репутация пользователей-авторов, хотя это и могло бы повысить качество классификатора.

    \subsection*{Кратко о последующих главах}

	В главе 1 производится обзор имеющихся решений и исследований по данной теме, их преимуществ и недостатков.

	В главе 2 описывается формальная постановка задачи, рассматриваются ее особенности и ставятся основные цели и задачи работы.

	В третьей главе описана работа по предварительной обработке и очистке данных с сайта StackOverflow, а также их анализ.

	Глава 4 содержит описание и детали реализации основных методов и подходов к поставленной задаче.

	В последней главе приведены результаты экспериментов с различными конфигурациями реализованных моделей, 
	а также проведен анализ полученных результатов и сделаны выводы о применимости модели к задаче определения правильного ответа.


\end{document}

